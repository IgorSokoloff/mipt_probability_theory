\documentclass[a4paper,12pt]{article}

%%% Работа с русским языком
\usepackage{cmap}					% поиск в PDF
\usepackage{mathtext} 				% русские буквы в формулах
\usepackage[T2A]{fontenc}			% кодировка
\usepackage[utf8]{inputenc}			% кодировка исходного текста
\usepackage[english,russian]{babel}	% локализация и переносы
\usepackage{comment}


%%% Дополнительная работа с математикой
\usepackage{amsfonts,amssymb,amsthm,mathtools} % AMS
\usepackage{amsmath}
\usepackage{icomma} % "Умная" запятая: $0,2$ --- число, $0, 2$ --- перечисление

%% Номера формул
%\mathtoolsset{showonlyrefs=true} % Показывать номера только у тех формул, на которые есть \eqref{} в тексте.

%% Шрифты
\usepackage{euscript}	 % Шрифт Евклид
\usepackage{mathrsfs} % Красивый матшрифт

\usepackage{extsizes} % Возможность сделать 14-й шрифт
\usepackage{geometry} % Простой способ задавать поля
\geometry{top=25mm}
\geometry{bottom=35mm}
\geometry{left=15mm}
\geometry{right=15mm}

\usepackage{chngcntr}
\usepackage{hyperref}

\usepackage{setspace} % Интерлиньяж
%\onehalfspacing % Интерлиньяж 1.5
%\doublespacing % Интерлиньяж 2
%\singlespacing % Интерлиньяж 1

\usepackage{lastpage} % Узнать, сколько всего страниц в документе.
\usepackage{soulutf8} % Модификаторы начертания

\counterwithin*{equation}{section}
\counterwithin*{equation}{subsection}



%% Свои команды
\DeclareMathOperator{\sgn}{\mathop{sgn}}

%% Перенос знаков в формулах (по Львовскому)
\newcommand*{\hm}[1]{#1\nobreak\discretionary{}
{\hbox{$\mathsurround=0pt #1$}}{}}

%%% Работа с картинками
\usepackage{graphicx}  % Для вставки рисунков
\graphicspath{{images/}{images2/}}  % папки с картинками
\setlength\fboxsep{3pt} % Отступ рамки \fbox{} от рисунка
\setlength\fboxrule{1pt} % Толщина линий рамки \fbox{}
\usepackage{wrapfig} % Обтекание рисунков и таблиц текстом

%%% Работа с таблицами
\usepackage{array,tabularx,tabulary,booktabs} % Дополнительная работа с таблицами
\usepackage{longtable}  % Длинные таблицы
\usepackage{multirow} % Слияние строк в таблице
\usepackage{graphicx}
\usepackage{fancyhdr}
\usepackage{hyperref}
\usepackage{booktabs}

\newcommand{\lt}{\left}
\newcommand{\rt}{\right}
\newcommand{\al}{\alpha}
\newcommand{\p}{\partial}
\newcommand{\D}{\Delta}
\newcommand{\fr}{\frac}
\newcommand{\dfr}{\dfrac}
\newcommand{\mbf}{\mathbf}
\newcommand{\ol}{\overline}
\newcommand{\bb}{\mathbb}
\newcommand{\om}{\Omega}
\newcommand{\Rw}{\Rightarrow}
\newcommand{\ve}{\varepsilon}
\newcommand{\vp}{\varphi}
\newcommand{\mc}{\mathcal}
\newcommand{\sg}{\sigma}


\pagestyle{fancy}
\fancyhf{}
\pagestyle{plain} % нумерация вкл.

\rhead{\today}
\lhead{Соколов Игорь, группа 573}

%%% Заголовок
\author{Соколов Игорь, группа 573}
\title{ДЗ №6 по Теории Вероятностей }
\date{\today}

\begin{document} % конец преамбулы, начало документа

\maketitle

\textbf {№ 104}

Пусть в вероятностном пространстве $(\Omega, \mc F, \bb P), \Omega = [0,1), \mc F \-- \sigma$-алгебра, содержащая полуинтервалы вида $\om_{in} = [(i-1)/n, i/n], i \in [1, n], n\in \bb N$  и $P$ - мера Лебега $(\forall i, n :P\{\omega \in \om_{in}\}= 1/n)$
Исследовать на сходимость послед случайных величин $X_1^{(1)},X_2^{(1)},X_2^{(2)},X_3^{(1)},X_3^{(2)},X_3^{(3)},...$
\begin{itemize}
\item \textbf{б)}
$X_n^{(i)}(\omega) = 
\begin{cases}
1/n,& \omega \in \om_{in}\\
0,& \omega \in \om\backslash\om_{in}
\end{cases}
$

	$X_n^{(i)}(\omega) = 
	\begin{cases}
	1/n,& p = 1/n\\
	0,& p = 1 - 1/n
	\end{cases}
	$

\begin{enumerate}
	\item в среднеквадратичном:
	$$\bb E|X_n|^2 = \fr{1}{n^2}\cdot\fr{1}{n} + 0\cdot \lt(1 - \fr{1}{n}\rt) = \fr{1}{n^3} \xrightarrow[n \to +\infty]{} 0 $$
	
	$\Rightarrow$
	$\exists$ сходимость в среднеквадратичном $\Rightarrow \exists$ сходимость по вероятности $\Rightarrow \exists$ сходимость по распределению
	\item почти наверное
	
	$$P\lt\{\omega\mid \fr{1}{n}\xrightarrow[n \to +\infty]{} 0 \rt\} = 1$$
	$\Rightarrow$ есть сходимость п.н.
\end{enumerate}

\item \textbf{в)} $X_n^{(i)}(\omega) = 
\begin{cases}
n,& \omega \in \om_{in}\\
1/n,& \omega \in \om\backslash\om_{in}
\end{cases}
$

$X_n^{(i)}(\omega) = 
\begin{cases}
n,& p = 1/n\\
1/n,& p =  1 - 1/n
\end{cases}
$

Видно, что $X_n\xrightarrow[n \to +\infty]{} X = \begin{cases}
1,& p = 0\\
0,& p =  1
\end{cases}$
\begin{enumerate}
\item по распределению:

Приведем следующие теоремы:
\begin{itemize}
\item\textit{$F_n \rightarrow F \Leftrightarrow \vp_n(t)\rightarrow \vp(t)$ при каждом $t$, где $\vp(t) - $х.ф., соответствующая $F$(сходимость по распределению эквивалентна сходимости характеристических функций)}

\item\textit{Пусть $\vp_n(t)$ - последовательность характеристических функций: $\vp_n(t)\xrightarrow[n \to +\infty]{}\vp(t) $ при каждом $t$. Тогда следующие условия эквивалентны:
\begin{enumerate}
	\item $\vp(t)$ является х.ф.;
	\item $\vp(t)$ непрерывна в точке $t = 0$.
\end{enumerate}
}
\end{itemize}

$\vp_{X_n}(t) = \bb E e^{itX_n} = e^{itn}\cdot\fr{1}{n} + e^{i\fr{t}{n}}\cdot\lt(1 - \fr{1}{n}\rt) \xrightarrow[n \to +\infty]{} 1 = \vp_X(t)$

$\vp_X(t)$ непрерывна в нуле $\Rightarrow$ $\vp_X(t)$ - х.ф.(по сформулированной выше теореме)

$\vp_X(t)$ есть х.ф. случайной величины $X = 0$ с вероятностью $p = 1$

$\Rightarrow$$\vp_X(t)$ есть х.ф. $F$

$\exists $ сходимость по распределению

Так как ранее было получено, что $X_n \rightarrow 0$(к константе), то в данном случае из сходимости по распределению будет следовать сходимость по вероятности.

\item почти наверное:
	$$P\lt\{\omega\mid X_n \rightarrow 0\rt\} = 1$$
$\nexists$ сходимость п.н.

\item в среднеквадратичном:
$\bb E|X_n - X|^2 = \bb E|X_n|^2 = n^2 \fr{1}{n} + \fr{1}{n}\lt(1 - \fr{1}{n}\rt)\xrightarrow[n \to +\infty]{} \infty$

$\nexists$ сходимость в среднеквадратичном.

\end{enumerate}

\item \textbf{г)}

$X_n^{(i)}(\omega) = 
\begin{cases}
1/n,& \omega \in \om_{in}\\
1 - 1/n,& \omega \in \om\backslash\om_{in}
\end{cases}
$

$X_n^{(i)}(\omega) = 
\begin{cases}
1/n,& p = 1/n\\
1 - 1/n,& p =  1 - 1/n
\end{cases}
$
\begin{enumerate}
\item по распределению:
Аналогично пользуемся аппаратом х.ф.

$\vp_{X_n}(t) = \bb E e^{itX_n} = e^{it\fr{1}{n}}\cdot\fr{1}{n} + e^{it\lt(1 - \fr{1}{n} \rt)}\cdot\lt(1 - \fr{1}{n}\rt) \xrightarrow[n \to +\infty]{} e^{it} = \vp_X(t)$

Получили х.ф. случайной величины $X = 1$ с вероятностью $p = 1$

Аналогично имеет место сходимость по распределению.

Так как $X_n$ сходится к константе, то $\exists$ сходимость по вероятности

\item в среднеквадратичном:
$\bb E|X_n - 1|^2  = \lt(1 - \fr{1}{n}\rt)^2\fr{1}{n} + \lt(1 - \fr{1}{n}\rt)\lt(\fr{1}{n}\rt)^2 = \lt(1 - \fr{1}{n}\rt)\fr{1}{n}\rightarrow 0$

$\exists$ сходимость в среднеквадратичном.

\item почти наверное:
$$P\lt\{\omega\mid X_n \rightarrow 1\rt\} = 1$$
$\exists$ сходимость п.н.
\end{enumerate}

\end{itemize}

\textbf {№ 121}

Случайные величины $X_1, \dots, X_n$ независимы и имеют распределение Коши
$$f(x) = \fr{d}{\pi(d^2 + x^2)}$$

Докажите равенство распределений $X_1$ и $\fr{X_1  и X_1 + X_2+ \dots + X_n}{n}$. Противоречит ли это ЗБЧ или ЦПТ?

\begin{proof}

ЗБЧ и ЦПТ требуют существование конечного первого момента, но распределение Коши не имеет моментов $\Rightarrow$ не может быть использовано в данном случае.

Докажем равенство распределений через равенство х.ф.

Известно, что х.ф. для распределение Коши выражается так:
$$\vp_{X_1} = e^{-d|t|}$$

$$\vp_{\fr{\sum\limits_{j=1}^{n}X_j}{n}} = \bb Ee^{i\fr{\sum\limits_{j=1}^{n}X_j}{n}t}  = \prod\limits_{j=1}^{n} \vp_{X_j}\lt(\fr{t}{n}\rt) = \lt[\vp_{X_1}\lt(\fr{t}{n}\rt)\rt]^n = \lt[e^{-d|\fr{t}{n}|}\rt]^n = e^{-d|t|} = \vp_{X_1}(t)$$

Второе и третье равенства верны так как случайные величины независимы и из одного распределения.

Получили равенство х.ф.
$\Rightarrow$ равенство распределений.
\end{proof}
\end{document} % конец документа

