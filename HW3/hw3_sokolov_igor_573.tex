\documentclass[a4paper,12pt]{article}

%%% Работа с русским языком
\usepackage{cmap}					% поиск в PDF
\usepackage{mathtext} 				% русские буквы в формулах
\usepackage[T2A]{fontenc}			% кодировка
\usepackage[utf8]{inputenc}			% кодировка исходного текста
\usepackage[english,russian]{babel}	% локализация и переносы
\usepackage{comment}


%%% Дополнительная работа с математикой
\usepackage{amsfonts,amssymb,amsthm,mathtools} % AMS
\usepackage{amsmath}
\usepackage{icomma} % "Умная" запятая: $0,2$ --- число, $0, 2$ --- перечисление

%% Номера формул
%\mathtoolsset{showonlyrefs=true} % Показывать номера только у тех формул, на которые есть \eqref{} в тексте.

%% Шрифты
\usepackage{euscript}	 % Шрифт Евклид
\usepackage{mathrsfs} % Красивый матшрифт

\usepackage{extsizes} % Возможность сделать 14-й шрифт
\usepackage{geometry} % Простой способ задавать поля
\geometry{top=25mm}
\geometry{bottom=35mm}
\geometry{left=20mm}
\geometry{right=20mm}

\usepackage{chngcntr}
\usepackage{hyperref}

\usepackage{setspace} % Интерлиньяж
%\onehalfspacing % Интерлиньяж 1.5
%\doublespacing % Интерлиньяж 2
%\singlespacing % Интерлиньяж 1

\usepackage{lastpage} % Узнать, сколько всего страниц в документе.
\usepackage{soulutf8} % Модификаторы начертания

\counterwithin*{equation}{section}
\counterwithin*{equation}{subsection}



%% Свои команды
\DeclareMathOperator{\sgn}{\mathop{sgn}}

%% Перенос знаков в формулах (по Львовскому)
\newcommand*{\hm}[1]{#1\nobreak\discretionary{}
{\hbox{$\mathsurround=0pt #1$}}{}}

%%% Работа с картинками
\usepackage{graphicx}  % Для вставки рисунков
\graphicspath{{images/}{images2/}}  % папки с картинками
\setlength\fboxsep{3pt} % Отступ рамки \fbox{} от рисунка
\setlength\fboxrule{1pt} % Толщина линий рамки \fbox{}
\usepackage{wrapfig} % Обтекание рисунков и таблиц текстом

%%% Работа с таблицами
\usepackage{array,tabularx,tabulary,booktabs} % Дополнительная работа с таблицами
\usepackage{longtable}  % Длинные таблицы
\usepackage{multirow} % Слияние строк в таблице
\usepackage{graphicx}
\usepackage{fancyhdr}
\usepackage{hyperref}
\usepackage{booktabs}

\newcommand{\lt}{\left}
\newcommand{\rt}{\right}
\newcommand{\al}{\alpha}
\newcommand{\p}{\partial}
\newcommand{\D}{\Delta}
\newcommand{\fr}{\frac}
\newcommand{\dfr}{\dfrac}
\newcommand{\mbf}{\mathbf}
\newcommand{\ol}{\overline}
\newcommand{\bb}{\mathbb}
\newcommand{\om}{\Omega}
\newcommand{\Rw}{\Rightarrow}


\pagestyle{fancy}
\fancyhf{}
\pagestyle{plain} % нумерация вкл.

\rhead{\today}
\lhead{Соколов Игорь, группа 573}

%%% Заголовок
\author{Соколов Игорь, группа 573}
\title{ДЗ по Теории Вероятностей к семинару №6.}
\date{\today}

\begin{document} % конец преамбулы, начало документа

\maketitle



\section{}

Показать, что если случайные величины $X_1$ и $X_2$ независимы и имеют нормальное распределение с математическими ожиданиями $\mu_1$ и $\mu_2$ и дисперсиями $\sigma_1^2$ и
$\sigma_2^2$ соответственно, то $X_1$ + $X_2$ также имеет нормальное распределение с математическим ожиданием $\mu_1$ + $\mu_2$ и дисперсией $\sigma_1^2$ и
$\sigma_2^2$ .

\vspace{\baselineskip}

\textbf{Решение:}

\begin{enumerate}

\item
Приведем доказательство линейности мат. ожидания в общем случае для произвольного распределения. $\Rightarrow$ для нормального это свойство выполнится.

$$EX_1 + EX_2 = \int\limits_\mathbb{R} xf_x(x)dx + \int\limits_\mathbb{R} yf_y(y)dy$$

где $f_x(x), f_y(y) -$ плотности распределения случайных величин $X_1, X_2$ соответственно

Рассмотрим функцию совместного распределения $F_{xy}(x,y) = P(x<a, y<b)$

Для непрерывного распределения верно: $f_x(x) = \int\limits_\mathbb{R} xf_{xy}(y, x)dy$ .

\item[$\Longrightarrow$]  
\begin{multline*}
EX_1+EX_2 = \int\limits_\mathbb{R} xf_x(x)dx + \int\limits_\mathbb{R} yf_y(y)dy = \int\limits_\mathbb{R} x \left[\int\limits_\mathbb{R} xf_{xy}(y, x)dy \right]dx \ +\\
+ \int\limits_\mathbb{R} y \left[\int\limits_\mathbb{R} xf_{xy}(x, y)dx \right]dy =  \int\limits_\mathbb{R}\int\limits_\mathbb{R} (x+y) f_{xy}(x, y)dxdy = E(X_1+X_2) 
\end{multline*}

$\Longrightarrow E(X_1+X_2) = EX_1 + EX_2 = \mu_1 +\mu_2$\\

\item[2)] Проведем абсолютно аналогичные рассуждения для дисперсии

\[ DX = EX^2 - (EX)^2 \]

\begin{multline*}
DX_1+DX_2 = \int\limits_\mathbb{R} x^2f_x(x)dx -\left(\int\limits_\mathbb{R} xf_x(x)dx\right)^2 + \int\limits_\mathbb{R} y^2f_y(y)dy -\left(\int\limits_\mathbb{R} yf_y(y)dy\right)^2 =\\
= \int\limits_\mathbb{R} x^2 \left[\int\limits_\mathbb{R} f_{xy}(y, x)dy \right]dx + \int\limits_\mathbb{R} y^2 \left[\int\limits_\mathbb{R} f_{xy}(x, y)dx \right]dy  - \int\limits_\mathbb{R} x \left[\int\limits_\mathbb{R} xf_{xy}(y, x)dy \right]dx
\end{multline*}

\end{enumerate}


\section{}

Показать, что если $\bb E|\xi|<\infty$, то $\forall a>0 \rightarrow P\lt(|\xi|\ge a \rt) \le \fr{\bb E|\xi|}{a}$

\begin{proof}
\begin{comment}

Пусть $A$ - некоторое событие и $\bb P(A) = p$

Введем индикаторную случайную величину:
$$\chi(\omega) = 
\begin{cases}
1& \omega \in A\\
0& \omega \notin A\\
\end{cases}$$

Случайная величина $\chi$ имеет распределение Бернулли с параметром $p$:
$$\bb P(\chi = 1) = \bb P(A) = p$$
и ее математическое ожидание равно вероятности успеха $p = \bb P(A)$.

Очевидно, что для индикаторов событий $A$ и $\ol A$ справедливо:
$$\chi$$
\end{comment}
$$\bb E |\xi| = \int\limits_{0}^{+\infty}xdF(x)\ge \int\limits_{a}^{+\infty}xdF(x) \ge a\int\limits_{a}^{+\infty}dF(x) = a\bb P(|\xi|\ge a)$$

$$\Rightarrow \fr{\bb E |\xi|}{a} \ge \bb P(|\xi|\ge a)$$
\end{proof}

%3

\section{}

Вычислить k-ый момент для стандартного($\mu = 0, \sigma^2=1$) нормального распределения.

\vspace{\baselineskip}

\textbf{Решение:}

$$
E\xi = \int \limits _ {-\infty}^{\infty} \dfrac{x^k}{\sqrt{2\pi}} \exp\left( \dfrac{-x^2}{2} \right)dx = \dfrac{1}{\sqrt{2\pi}} \int \limits _ {-\infty}^{\infty} -x^{k-1} \cdot d \left[ \exp \left( \dfrac{-x^2}{2} \right) \right] \overrightarrow{\small{\text{(по частям)}}}=$$ $$= \dfrac{1}{\sqrt{2\pi}}  (-x^{k-1}) \cdot \exp \left( \dfrac{-x^2}{2} \right) \Bigg|^{\infty} _{-\infty} -\quad \dfrac{1}{\sqrt{2\pi}}\int \limits _ {-\infty}^{\infty} (k-1)(-x^{k-2}) \cdot  \exp \left( \dfrac{-x^2}{2} \right) dx = 
$$
\begin{multline}\label{eq_1}
\\  = \dfrac{1}{\sqrt{2\pi}}  (-x^{k-1}) \cdot \exp \left( \dfrac{-x^2}{2} \right) \Bigg|^{\infty} _{-\infty} - \quad \dfrac{1}{\sqrt{2\pi}} \int \limits ^{\infty}_{-\infty} (k-1) (x^{k-3}) \cdot d \left[\exp \left( \dfrac{-x^2}{2} \right)\right]  \overrightarrow{\small{\text{(по частям)}}} \\ \cdots \text{и так далее}
\end{multline}

В зависимости от числа $k$, будем получать разные ответы:
\begin{enumerate}
	\item[1)] $k$-нечетное:\\
	Тогда \eqref{eq_1} =0, так как все члены четные.
	\item[2)] $k$ -четное:\\
	Тогда все члены без интегралла уйдут(они равны 0).
	
	Получаем:
	 $$\dfrac{-1}{\sqrt{2\pi}} \cdot \int \limits _ {-\infty}^{\infty} (k-1)(k-3) \cdots  d \left[ \exp \left( \dfrac{-x^2}{2} \right) \right]  = {\sigma^p (k-1)!!}$$
\end{enumerate}

%4
\section{}

Пусть $X_n$ - количество чисел, не изменивших свои позиции при случайной перестановке элементов множества ${1,\dots , n}$. Найдите дисперсию $X_n$.

\vspace{\baselineskip}

\textbf{Решение:}

\vspace{\baselineskip}

Пусть

$\Omega - $  множество всех равновероятных перестановок.

$\xi_i(\omega) = \begin{cases}
1 &\text{$i$-ый элемент остался на месте}\\
0 &\text{$i$-ый элемент поменял свою позицию}
\end{cases}$
$$\Rightarrow $$

$\Rw \Omega$ состоит из всевозможных последовательностей длины $n$ из нулей и единиц.

Число перестановок, в которых $i$-ый элемент остался на месте - $(n-1)!$.

$$\Rightarrow \bb P(xi_i = 1) = \fr{(n-1)!}{n!} = \fr{1}{n}$$

Видно что введенная нами случайная величина имеет Биномиальное распределение.

Тогда имеем $$\bb E\xi = np = n\cdot\fr{1}{n}$$ где $p$ - вероятность успеха.

$$\bb D\xi = npq = np(1 - p) = \fr{n - 1}{n}$$

\textbf{Ответ:} $\bb D\xi = \dfr{n - 1}{n}$

\section{}

Какова вероятность того, что в случайном графе на 4 вершинах с вероятностью
проведения ребра $\fr{2}{3}$ нет треугольников?

\textbf{Решение:}

\vspace{\baselineskip}

$n = 4$ по условию.

$V = \{V_1,\dots,V_n\}$

$E = \{e_1,\dots,e_{C^2_n}\}$

$$\bb P\{\text{в графе $k$ ребер }\} = \bb P\{|E| = k\} = C_{C^2_n}^k p^k(1-p)^{{C^2_n} - k}$$

$C_n^3 -$ число <<троек>> ребер в графе.

$$\bb P\{\text{в графе $k$ треугольников }\} = C_{C^3_n}^k p^{3k}(1-p)^{{C^3_n} - 3k}$$

$$\Rw \bb P\{\text{в графе $0$ треугольников }\} = (1 - p)^{C^3_4} = \lt(1 - \fr{2}{3}\rt)^4 = \fr{1}{3^4} = 0.012$$

\textbf{Ответ:} 0.012

\end{document} % конец документа

