\documentclass[a4paper,12pt]{article}

%%% Работа с русским языком
\usepackage{cmap}					% поиск в PDF
\usepackage{mathtext} 				% русские буквы в формулах
\usepackage[T2A]{fontenc}			% кодировка
\usepackage[utf8]{inputenc}			% кодировка исходного текста
\usepackage[english,russian]{babel}	% локализация и переносы
\usepackage{comment}


%%% Дополнительная работа с математикой
\usepackage{amsfonts,amssymb,amsthm,mathtools} % AMS
\usepackage{amsmath}
\usepackage{icomma} % "Умная" запятая: $0,2$ --- число, $0, 2$ --- перечисление

%% Номера формул
%\mathtoolsset{showonlyrefs=true} % Показывать номера только у тех формул, на которые есть \eqref{} в тексте.

%% Шрифты
\usepackage{euscript}	 % Шрифт Евклид
\usepackage{mathrsfs} % Красивый матшрифт

\usepackage{extsizes} % Возможность сделать 14-й шрифт
\usepackage{geometry} % Простой способ задавать поля
\geometry{top=25mm}
\geometry{bottom=35mm}
\geometry{left=15mm}
\geometry{right=15mm}

\usepackage{chngcntr}
\usepackage{hyperref}

\usepackage{setspace} % Интерлиньяж
%\onehalfspacing % Интерлиньяж 1.5
%\doublespacing % Интерлиньяж 2
%\singlespacing % Интерлиньяж 1

\usepackage{lastpage} % Узнать, сколько всего страниц в документе.
\usepackage{soulutf8} % Модификаторы начертания

\counterwithin*{equation}{section}
\counterwithin*{equation}{subsection}



%% Свои команды
\DeclareMathOperator{\sgn}{\mathop{sgn}}

%% Перенос знаков в формулах (по Львовскому)
\newcommand*{\hm}[1]{#1\nobreak\discretionary{}
{\hbox{$\mathsurround=0pt #1$}}{}}

%%% Работа с картинками
\usepackage{graphicx}  % Для вставки рисунков
\graphicspath{{images/}{images2/}}  % папки с картинками
\setlength\fboxsep{3pt} % Отступ рамки \fbox{} от рисунка
\setlength\fboxrule{1pt} % Толщина линий рамки \fbox{}
\usepackage{wrapfig} % Обтекание рисунков и таблиц текстом

%%% Работа с таблицами
\usepackage{array,tabularx,tabulary,booktabs} % Дополнительная работа с таблицами
\usepackage{longtable}  % Длинные таблицы
\usepackage{multirow} % Слияние строк в таблице
\usepackage{graphicx}
\usepackage{fancyhdr}
\usepackage{hyperref}
\usepackage{booktabs}

\newcommand{\lt}{\left}
\newcommand{\rt}{\right}
\newcommand{\al}{\alpha}
\newcommand{\p}{\partial}
\newcommand{\D}{\Delta}
\newcommand{\fr}{\frac}
\newcommand{\dfr}{\dfrac}
\newcommand{\mbf}{\mathbf}
\newcommand{\ol}{\overline}
\newcommand{\bb}{\mathbb}
\newcommand{\om}{\Omega}
\newcommand{\Rw}{\Rightarrow}
\newcommand{\ve}{\varepsilon}
\newcommand{\vp}{\varphi}


\pagestyle{fancy}
\fancyhf{}
\pagestyle{plain} % нумерация вкл.

\rhead{\today}
\lhead{Соколов Игорь, группа 573}

%%% Заголовок
\author{Соколов Игорь, группа 573}
\title{ДЗ по Теории Вероятностей к семинару №8.}
\date{\today}

\begin{document} % конец преамбулы, начало документа

\maketitle



\section{}

Доказать следующие утверждения:

\begin{enumerate}
\item х.ф - равномерно непрерывная функция
\begin{proof}
Нам понадобится \textbf{теорема о мажорируемой последовательности}:
	
{\it Если $\xi_n\xrightarrow[p]{}\xi$ и $|\xi_n|<\eta, \bb E\eta <\infty$, то существует $\bb E \xi$ и $\bb E|\xi_n - \xi|\rightarrow 0$ }
	
\begin{multline}
|\varphi(t + h) - \varphi(t)| = |\bb E(e^{i(t+h)\xi} - e^{it\xi})| = |\bb E[(e^{it\xi})(e^{ih\xi} - 1)]| = |\bb E(e^{it\xi})\bb E(e^{ih\xi} - 1)| \le |\bb E(e^{ih\xi} - 1)|\rightarrow 0 
\end{multline}
по теореме о мажорируемой последовательности так как 

$|e^{ih\xi} - 1| \xrightarrow[p]{} 0$ при $h \rightarrow 0, |e^{ih\xi} - 1)|\le 2$
\end{proof}		

\item $\ol \varphi_{\xi}(t) = \varphi_{\xi}(-t) = \varphi_{-\xi}(t)$
\begin{proof}
$\ol \varphi_{\xi}(t) = \ol{\bb Ee^{it\xi}} = \bb E\ol{e^{it\xi}} = \bb Ee^{-it\xi} = \varphi_{\xi}(-t) = \varphi_{-\xi}(t)$
\end{proof}
\item 
Если х.ф. интегрируема на всей действительной оси, то существует плотность
соответствующей случайной величины и выполнено:
$$f(x) = \fr{1}{2\pi} \int\limits_{-\infty}^{+\infty}e^{-itx}\varphi(t)dt$$
\begin{proof}
	\begin{enumerate}
	\item
	Сначало установим вспомогательное равенство:
	\begin{equation}\label{ref_1}
	\forall \varepsilon \rightarrow  p_{\ve}(t)\ = \fr{1}{2\pi}\int\limits_{-\infty}^{+\infty}e^{itu}\vp(u)e^{-\ve^2u^2/2}du \equiv \fr{1}{\sqrt{2\pi}\ve}\int\limits_{-\infty}^{+\infty}\exp\lt\{-\fr{(u - t)^2}{2\ve^2} \rt\}F(du)
	\end{equation}
	где $F$ - распределение $\xi$.
	
	Исходным является равенство
	\begin{equation}\label{ref_2}
	\fr{1}{\sqrt{2\pi}}\int\limits_{-\infty}^{+\infty}\exp\lt\{ix\fr{\xi - t}{\ve}  - \fr{x^2}{2}\rt\}dx = \exp\lt\{-\fr{(\xi - t)^2}{2\ve^2} \rt\}
	\end{equation}
	
	В обеих частях стоит х.ф. стандартного нормального распределения в точке $(\xi - t)/\ve$. Сделав замену $x = \ve u$, левая часть этого равенства можно записать в виде 
	
	$$\fr{\ve}{\sqrt{2\pi}}\int\limits_{-\infty}^{+\infty}\exp\lt\{iu (\xi - t)  - \fr{\ve^2u^2}{2}\rt\}du$$
	
	Взяв, математическое ожидание от обеих частей \eqref{ref_2}, получаем
	
	$$\fr{\ve}{\sqrt{2\pi}}\int\limits_{-\infty}^{+\infty}e^{-itu}\vp(u)e^{-\ve^2u^2/2}du = \int\limits_{-\infty}^{+\infty}\exp\lt\{-\fr{(u - t)^2}{2\ve^2} \rt\}F(du)$$
	
	Что доказывает равенство \eqref{ref_1}.
	
	\item Для доказательства исходной теоремы рассмотрим сначала левую часть равенства \eqref{ref_1}. Так как $e^{-\ve^2u^2/2}\rightarrow 1$ при $\ve \rightarrow 0, |e^{-\ve^2u^2/2}|\le 1$ и $\vp(u)$ интегрируема, то 
	\begin{equation}\label{ref_4}
	p_{\ve}(t) = \fr{1}{2\pi}\int\limits_{-\infty}^{+\infty}e^{-itu}\vp(u)du = p_0(t)
	\end{equation}
	при $\ve \rightarrow 0$ равномерно по $t$, так как интеграл в левой части \eqref{ref_1} равномерен по $t$. 
	
	Отсюда также следует, что 
	\begin{equation}\label{ref_5}
	\int\limits_{a}^{b} p_{\ve}(t)dt \rightarrow \int\limits_{a}^{b} p_{0}(t)dt.
	\end{equation}
	
	Рассмотрим левую часть равенства\eqref{ref_1}. Она представляет собой суммы $\xi + \ve \eta$, где $\xi$ и $\eta$ независимы и $\eta$ из стандартного нормального распределения. Поэтому
	\begin{equation}\label{ref_6}
	\int\limits_{a}^{b} p_{\ve}(t)dt = \bb P (a < \xi + \ve\eta\le b).
	\end{equation}
	
	Так как $\xi + \ve\eta \xrightarrow[p]{} \xi$ при $\ve \rightarrow 0$ и $\forall a, b$ (фиксированных) предел $\int\limits_{a}^{b} p_{\ve}(t)dt$ существует по полученному следствию \eqref{ref_5}, то  \eqref{ref_6} может быть только $F\lt([a, b)\rt)$
	
	Из \eqref{ref_5} и \eqref{ref_6} получаем
	\begin{equation}\label{ref_7}
		\int\limits_{a}^{b} p_{0}(t)dt = F\lt([a, b)\rt)
	\end{equation}
	
	Получаем, что распределение $F$ имеет плотность $p_0(t)$ определяемую соотношением \eqref{ref_4}.
	
	Ограниченность $p_0(t)$ следует из интегрируемости $\vp$:
	\begin{equation}
	p_0(t) \le \fr{1}{2\pi}\int\limits_{-\infty}^{+\infty}|\vp(t)|dt < \infty
	\end{equation}
	
\end{enumerate}
	
\end{proof}	

\end{enumerate}	


%2

\section{}

Является ли $f(t)$ характеристической функцией
\begin{enumerate}
\item $f(t) = \exp(-t^4)$ 
\begin{proof}
Предположим, что $f(t)$ является характеристической.
	
Воспользуемся следующими теоремами из Боровкова (стр 152 -153):
\begin{itemize}
	\item \textit{Если существует конечный $k$-й момент($k \ge 1$), то существует непрерывная $k$-я производная $\vp^{(k)}(0)$ и $\vp^{(k)}(0) = i^k \bb E\xi^k$}
	\item \textit{Если производная четного $\vp^{(2k)}$ существует, то $\bb E |\xi^{2k}| < \infty$ и $\vp^{(2k)} = (-1)^k\bb E \xi^{2k} $} 
\end{itemize}
	
Таким образом $\bb E\xi = -4t^3\exp(-t^4)\Big|_{t = 0} = 0$

И $\bb D\xi = \bb E(\xi - \bb E\xi)^2 = \bb E\xi^2 = 0$

Такие мат ожидание и дисперсия соответствуют случайной величине 
$\xi = 
\begin{cases} 
0,& p = 1\\
1,& p = 0
\end{cases}$ 

Но для $\vp_{\xi}(t) = \bb Ee^{it\xi} = \bb E1 = 1 \neq f(t)$

Получили противоречие.

$\Rightarrow f(t)$ \textbf{не является характеристической}.
\end{proof}	
\item $f(t) = \dfr{1}{1+t}$
\begin{proof}
Это действительная функция, но она не является чётной. Свойство $\ol \varphi_{\xi}(t) = \varphi_{\xi}(-t) = \varphi_{-\xi}(t)$ не выполнено. 

$\Rightarrow f(t)$ \textbf{не является характеристической}.
\end{proof}
\item $f(t) = \dfr{1}{1+it^2}$
\begin{proof}
$$f(t) = \dfr{1}{1+it^2} = \dfr{1-it^2}{1+t^4} = \dfr{1}{1+t^4} - i\dfr{t^2}{1+t^4}$$
$$ \ol f(t) = \dfr{1}{1+t^4} + i\dfr{t^2}{1+t^4}$$

Видно, что  $\ol f(t) \neq f(-t)$

$\Rightarrow f(t)$ \textbf{не является характеристической}.
\end{proof}

\item $f(t) = 1+\cos t$
\begin{proof}
При $t = 0 \rightarrow f(0) = 1 + \cos(0) = 2 >1$, что противоречит свойству $\forall t\rightarrow |f(t)|\le 1$

$\Rightarrow f(t)$ \textbf{не является характеристической}. 
\end{proof}	

\item $f(t) = \fr{1}{2}\cos^2t2^{\cos t}$

\item $f(t) = 
\begin{cases}
1 - t& |t| \le 1\\
0& \text{иначе}
\end{cases}$
\begin{proof}

\begin{itemize}
	\item Функция действительного переменного, не являющаяся четной;
	\item Разрыв в точке $-t \Rightarrow$ не является равномерно непрерывной.
\end{itemize}

$\Rightarrow f(t)$ \textbf{не является характеристической}.

\end{proof}

\end{enumerate}

\end{document} % конец документа

