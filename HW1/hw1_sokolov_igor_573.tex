\documentclass[a4paper,12pt]{article}

%%% Работа с русским языком
\usepackage{cmap}					% поиск в PDF
\usepackage{mathtext} 				% русские буквы в формулах
\usepackage[T2A]{fontenc}			% кодировка
\usepackage[utf8]{inputenc}			% кодировка исходного текста
\usepackage[english,russian]{babel}	% локализация и переносы
\usepackage{comment}


%%% Дополнительная работа с математикой
\usepackage{amsfonts,amssymb,amsthm,mathtools} % AMS
\usepackage{amsmath}
\usepackage{icomma} % "Умная" запятая: $0,2$ --- число, $0, 2$ --- перечисление

%% Номера формул
%\mathtoolsset{showonlyrefs=true} % Показывать номера только у тех формул, на которые есть \eqref{} в тексте.

%% Шрифты
\usepackage{euscript}	 % Шрифт Евклид
\usepackage{mathrsfs} % Красивый матшрифт

\usepackage{extsizes} % Возможность сделать 14-й шрифт
\usepackage{geometry} % Простой способ задавать поля
\geometry{top=25mm}
\geometry{bottom=35mm}
\geometry{left=20mm}
\geometry{right=20mm}

\usepackage{chngcntr}
\usepackage{hyperref}

\usepackage{setspace} % Интерлиньяж
%\onehalfspacing % Интерлиньяж 1.5
%\doublespacing % Интерлиньяж 2
%\singlespacing % Интерлиньяж 1

\usepackage{lastpage} % Узнать, сколько всего страниц в документе.
\usepackage{soulutf8} % Модификаторы начертания

\counterwithin*{equation}{section}
\counterwithin*{equation}{subsection}



%% Свои команды
\DeclareMathOperator{\sgn}{\mathop{sgn}}

%% Перенос знаков в формулах (по Львовскому)
\newcommand*{\hm}[1]{#1\nobreak\discretionary{}
{\hbox{$\mathsurround=0pt #1$}}{}}

%%% Работа с картинками
\usepackage{graphicx}  % Для вставки рисунков
\graphicspath{{images/}{images2/}}  % папки с картинками
\setlength\fboxsep{3pt} % Отступ рамки \fbox{} от рисунка
\setlength\fboxrule{1pt} % Толщина линий рамки \fbox{}
\usepackage{wrapfig} % Обтекание рисунков и таблиц текстом

%%% Работа с таблицами
\usepackage{array,tabularx,tabulary,booktabs} % Дополнительная работа с таблицами
\usepackage{longtable}  % Длинные таблицы
\usepackage{multirow} % Слияние строк в таблице
\usepackage{graphicx}
\usepackage{fancyhdr}
\usepackage{hyperref}
\usepackage{booktabs}

\newcommand{\lt}{\left}
\newcommand{\rt}{\right}
\newcommand{\al}{\alpha}
\newcommand{\ol}{\overline}
\newcommand{\bb}{\mathbb}
\newcommand{\bs}{\backslash}

\pagestyle{fancy}
\fancyhf{}
\pagestyle{plain} % нумерация вкл.

\rhead{\today}
\lhead{Соколов Игорь, группа 573}

%%% Заголовок
\author{Соколов Игорь, группа 573}
\title{ДЗ к семинару №3-4 по Теории Вероятностей.}
\date{\today}

\begin{document} % конец преамбулы, начало документа

\maketitle



\section{}

Показать, что из независимости событий $A$ и $B$ следует независимость событий
$\ol{A}$ и $B$, $A$ и $\ol{B}$, $\ol{A}$ и $\ol{B}$.

\vspace{\baselineskip}

\begin{proof}

Обозначение: $A\cap B = AB$

По условию : $\bb P(AB) = \bb P(A)\bb P(B)$

1) Из свойств вероятности имеем: 
$$ \bb P(A\cup B) = \bb P(A) +\bb P(B) - \bb P(AB) = \bb P(A) +\bb P(B) - \bb P(A) \bb P(B) $$
$$A \cup B = A \cup (B \bs AB)$$
$$\Rightarrow \bb P(A\cup B) = \bb P( A \cup (B \bs AB)) = \bb P(A) + \bb P(B \bs AB)-\bb P(AB \bs AB) $$
$$\bb P(AB \bs AB) =\bb P(\varnothing) = 0 $$ 
$$\Rightarrow \bb P(A\cup B) = \bb P(A) + \bb P(B \bs AB) = \bb P(A) +\bb P(B) - \bb P(AB) $$
Доказали, что $$\bb P(B \bs AB) = \bb P(B) - \bb P(AB)$$

\begin{multline}
	\bb P(\ol A B) = \bb P((\Omega \bs A) B) = \bb P(\Omega B\bs AB) = \bb P(B\bs AB) =  \bb P(B) - \bb P(AB) = \bb P(B)(1 - \bb P(A)) = \bb P(\ol A)\bb P(B)
\end{multline}

2) Аналогично:
\begin{multline}
\bb P(A \ol B) = \bb P(A(\Omega \bs B)) = \bb P(\Omega A\bs AB) = \bb P(A\bs AB) =  \bb P(A) - \bb P(AB) = \bb P(A)(1 - \bb P(B)) = \bb P(A)\bb P(\ol B)
\end{multline}

3) 
\begin{multline}
\bb P(\ol A \ol B) = \bb P(\ol A(\Omega \bs B)) = \bb P(\Omega \ol A\bs \ol A B) = \bb P(\ol A\bs \ol A B) =  \bb P(\ol A) - \bb P(\ol A B) = \bb P(\ol A)(1 - \bb P(B)) = \bb P(\ol A)\bb P(\ol B)
\end{multline}

При доказательстве использовали $\bb P(\ol A B) = \bb P(\ol A)P( B)$

\end{proof}
\vspace{\baselineskip}


\section{}

Подбрасываются три игральные кости. События $A$, $B$ и $C$ означают выпадение одинакового числа очков (соответственно) на первой и второй, на второй и
третьей, на первой и третьей костях. Являются ли эти события независимыми

a)Попарно.

b)В совокупности.

\vspace{\baselineskip}

\textbf{Решение:}

\vspace{\baselineskip}

$\Omega = \lt\{(a_1,a_2,a_3), a_i \in [1,6], i \in [1,3]\rt\}$ - цепочки из трех элементов, где $i - $ номер кости.

$|\Omega| = 6^3$  

\textbf{a)}

В событие $A$ у нас 6 способов задать $a_1$ = $a_2$ и 6 способов задать $a_3$

(считаем, что на 3-й кости может быть одинаковое число очков с первой и второй. Условие это не запрещает)

$$\bb P(A) = \dfrac{6^2}{6^3} = \dfrac{1}{6} $$

Аналогично $$\bb P(B) = \dfrac{6^2}{6^3} = \dfrac{1}{6} $$

Событие $AB$ означает выпадение одинакового числа на всех трех кубиках - 6 способов задать $a_1 = a_2 = a_3$

$$\bb P(AB) = \dfrac{6}{6^3} = \dfrac{1}{6^2} $$

$$\Rightarrow\bb P(AB) = \dfrac{1}{6^2} = \dfrac{1}{6}\cdot\dfrac{1}{6} =\bb P(A) \bb P(B)$$

Аналогично можно показать, что 

$$\bb P(AС) = \bb P(A) \bb P(C)$$
$$\bb P(BС) = \bb P(B) \bb P(C)$$

$\Rightarrow $ \textbf{события попарно независимы}

\textbf{b)}

Событие $ABC$ означает выпадение одинакового числа на всех трех кубиках - 6 способов задать $a_1 = a_2 = a_3$

$$\bb P(ABC) = \dfrac{6}{6^3} = \dfrac{1}{6^2} \neq \dfrac{1}{6}\cdot\dfrac{1}{6}\cdot\dfrac{1}{6} = \bb P(A)\bb P(B) \bb P(C)$$

$\Rightarrow $ \textbf{отсутствует независимость в совокупности}

%3

\section{}

Известно, что в результате бросания десяти игральных костей выпала по крайней мере две «пятёрки». Какова вероятность того, что число выпавших «пятёрок» больше пяти?

\vspace{\baselineskip}

\textbf{Решение:}

\vspace{\baselineskip}

$\Omega = \lt\{(a_1,\dots,a_10), a_i \in [1,6], i \in [1,10]\rt\}$ - последовательности чисел из 10 элементов, где $i - $ номер кости.

$|\Omega| = 6^{10}$ 

Введем события:

$A = \{\text{выпало 2 пятерки}\}$

$B = \{\text{выпало больше 5 пятерок}\}$

$C_i = \{\text{выпало } i \text{ пятерок}\}$

Надо найти:

$$\bb P(B|A) = \dfrac{\bb P(AB)}{\bb P(A)} = \dfrac{\bb P(B)}{\bb P(A)} = \dfrac{1- \bb P(\ol B)}{\bb P(A)}$$

где $\ol B = \{\text{выпало меньше или равно 5 пятерок}\}$.

\vspace{\baselineskip}


Рассмотрим знаменатель:

$C_{10}^2$ - количество способов зафиксировать две пятерки в последовательности из 10 элементов.

$5^8$ - количество способов задать все остальные элементы (имеем в виду, что "5" больше не встречается)

Получаем: $\bb P(A) = \dfrac{C_{10}^2 \cdot5^8}{6^{10}}$

\vspace{\baselineskip}

Рассмотрим числитель:

$\bb P( \ol B) = \bb P(C_0) + \dots +  \bb P(C_5)$

Аналогичными рассуждениями получаем:

$\bb P( \ol B) = \dfrac{C_{10}^0 5^{10}}{6^{10}} + \dfrac{C_{10}^1 5^{9}}{6^{10}}+ \dots + \dfrac{C_{10}^5 5^{5}}{6^{10}} = 0.9976$

$$\bb P(B|A) = \dfrac{0.0024}{0.291} = 0.0083$$

\vspace{\baselineskip}


\textbf{Ответ:} $\bb P(B|A) = 0.0083$

%4
\section{}

За тремя дверями находится две козы и автомобиль. Вы выбираете одну дверь,
затем ведущий открывает другую дверь, за которой находится коза. После этого
вы можете поменять свой выбор. Найти вероятность того, что вы выиграете, если
измените выбор.

\vspace{\baselineskip}

\textbf{Решение:}

\textbf{1)}Будем считать, что ведущий знает, где находится автомобиль.

\vspace{\baselineskip}

Пронумеруем двери для удобства. На вероятностном пространстве это не отразится.

Пусть вначале я выбрал дверь 1.

Ведущий открыл 3.

Рассмотрим следующие события:

$Y_i = \{\text{я выбираю дверь }i\} $

$H_i = \{\text{ведущий открывает дверь }i\}$

$C_i = \{\text{автомобиль находится за дверью }i\} $

$G_i = \{\text{коза находится за дверью }i\} $

\vspace{\baselineskip}


Из условия задачи понятно, что надо подсчитать вероятность $\bb P(C_2|Y_1H_3G_3)$.

Найдем вначале $$\bb P(C_1|Y_1H_3G_3) = \dfrac{\bb P(C_1H_3G_3|Y_1)\bb P(Y_1)}{\bb P(H_3G_3|Y_1)\bb P(Y_1)} = \dfrac{\bb P(C_1H_3G_3|Y_1)}{\bb P(H_3G_3|Y_1)}$$

Так как ведущий знает, где находится автомобиль, то $H_3 \subseteq G_3$

(он открывает дверь 3 только если за ней коза))
и $ G_1H_3Y_1 = G_1G_3Y_1$

(то есть если я выбираю дверь 1, за которой находится коза, то он открывает дверь 3 всякий раз, когда за ней находится коза)

Также имеем $\bb P(C_1|Y_1) = 1/3$ и $\bb P(H_3|Y_1C_1) = 1/2$ в силу симметрии.

В этом случае $\bb P(C_1H_3G_3|Y_1) = \bb P(C_1H_3|Y_1) = \bb P(C_1 Y_1) \bb P(H_3|Y_1C_1) = \dfrac{1}{3}\cdot\dfrac{1}{2} = \dfrac{1}{6}$.

С другой стороны $\bb P(H_3G_3|Y_1) = \bb P(H_3|Y_1) = \bb P((G_1 + C_1)H_3|Y_1) = \bb P(G_1H_3|Y_1) + \bb P(C_1H_3|Y_1)=\\=\dfrac{\bb P(G_1H_3Y_1)}{\bb P(Y_1)} + \dfrac{1}{6} = \dfrac{\bb P(G_1G_3Y_1)}{\bb P(Y_1)} + \dfrac{1}{6} = \bb P(C_2|Y_1)+\dfrac{1}{6} = \dfrac{1}{3}+\dfrac{1}{6} = \dfrac{1}{2}$

Таким образом $\bb P(C_1|Y_1H_3G_3) = \dfrac{1/6}{1/2} = \dfrac{1}{3}$

$\bb P(C_2|Y_1H_3G_3) = 1 - P(C_2|Y_1H_3G_3) = 1 - \dfrac{1}{3} = \dfrac{2}{3}$

Получаем, что если я изменю решение, то вероятность возрастет.

\vspace{\baselineskip}

\textbf{2)} Предположим, ведущий не знает, где находится автомобиль.

Тогда события $C_1, Y_1 $ и $H_3$ независимы и в силу симметрии получаем:

$\bb P(C_1|Y_1H_3G_3) = \bb P(C_2|Y_1H_3G_3) = \dfrac{1}{2}$

\textbf{Ответ:} \textbf{1)} 2/3; \textbf{2)} 1/2

%5
\section{}
На заводе, изготавливающем болты, на долю машин $A$, $B$, $C$ приходятся соответственно 25, 35, 40\% изделий. В их продукции брак составляет соответственно
5, 4, 2\%. Случайно выбранный болт оказался деффектным. Каковы вероятности
того, что он был изготовлен соответственно на машине $A$, $B$, $C$?

\vspace{\baselineskip}

\textbf{Решение:}

Рассмотрим следующие события:

$A = \{\text{болт изготовлен на машине А}\}$

$B =\{\text{болт изготовлен на машине B}\}$

$C = \{\text{болт изготовлен на машине C}\}$

$D = \{\text{встретился бракованный болт}\}$

Из условий имеем

$\bb P(D|A) = 0.05$

$\bb P(D|B) = 0.04$

$\bb P(D|C) = 0.02$

$\bb P(A) = 0.25$

$\bb P(B) = 0.35$

$\bb P(C) = 0.40$



Напишем формулу полной вероятности:
\[ P(A|D) = \dfrac{P(D|A)\cdot P(A)}{P(D|A)\cdot P(A) + P(D|B)\cdot P(B) + P(D|C)\cdot P(C)}\]
Знаменатель равен : $0.05\cdot 0.25 + 0.04\cdot 0.35 + 0.02 \cdot  0.40 = \dfrac{69}{2000}$


1. Для машины А:
	\[P(A|D) = \dfrac{0.05\cdot 0.25}{69/2000}  = 0.36\]

2. Для машины B:
	\[P(A|D) = \dfrac{0.04\cdot 0.35}{69/2000}  = 0.41\]

3. Для машины C:
	\[P(A|D) = \dfrac{0.02 \cdot  0.40}{69/2000}  = 0.23\]


\vspace{\baselineskip}




\end{document} % конец документа

